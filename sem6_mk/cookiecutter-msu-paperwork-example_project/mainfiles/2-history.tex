%!TEX root = ../graduate-work.tex

\section{Историческая справка}

\subsection{История возникновения}

Интегрирование как операция восходит к античной математике, где её использовали
для нахождения площадей, объёмов и других величин. Уже в трудах Архимеда можно
встретить методы, похожие на интегрирование, но в более примитивной форме. В
эпоху Возрождения и в XVII-XVIII веках математики начали формализовать понятие
интеграла. Особенно важным был вклад Исаака Ньютона и Готфрида Лейбница,
которые независимо друг от друга разработали основы дифференциального и
интегрального исчисления.

Тем не менее, проблема точного и формального определения интеграла оставалась
актуальной. До XIX века существовали разные подходы к нахождению площадей и
объёмов, но их формулировки не имели строгой математической основы.

\subsection{Идея Римана}

В середине XIX века Бернард Риман предложил новую концепцию интеграла, которая
стала общепринятой в математике. Его идея заключалась в следующем: для
вычисления интеграла функции, заданной на отрезке
$[a, b]$, нужно разбиение этого отрезка на небольшие части (сегменты), на
которых функция приближенно представляется простыми величинами
(например, прямыми отрезками или прямоугольниками).

Риман предложил делить отрезок $[a, b]$ на $n$ частей, вычислять сумму
произведений высоты функции на ширину каждого интервала, а затем исследовать
поведение этой суммы, когда разбиение становится всё более тонким. Если такая
сумма сходится к определённому значению при бесконечно мелком разбиении, то
говорят, что функция интегрируема в смысле Римана.

\subsection{Развитие и расширения}

Позднее, на рубеже XIX-XX веков, математики начали развивать и обобщать понятие
интеграла. Одним из значительных шагов стало введение обобщённых понятий
интегралов, таких как интеграл Лебега, который позволяет интегрировать более
широкие классы функций, включая те, которые не являются интегрируемыми
по Риману. Эти обобщения стали важными для теории вероятностей и других
областей математики.

Тем не менее, интеграл Римана остаётся важным инструментом для понимания
основной идеи интеграции и продолжает использоваться во многих областях,
включая физику, инженерию и экономику.

