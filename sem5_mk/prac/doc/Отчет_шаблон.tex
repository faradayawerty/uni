\documentclass[14pt,Otchet]{diplomwork}
\reporttype{по производственной практике}

\usepackage{fancyvrb}
\usepackage{comment}

\date{2023}
\author{МКб-3301-51-00}{%
Заботин Михаил Алексеевич
}


\napravlenie{02.03.01}{Математика и компьютерные науки}
\profile{Математические основы компьютерных наук}
\advisor{к. ф.-м. н., доцент}{Чупраков Д.\,В.}
\kafedra{фундаментальной математики}{Е.\,М.~Вечтомов}
\department{компьютерных и физико-математических наук}{Н.\,А.~Бушмелева}
\institute{математики и информационных систем}
%\practicsbase{ФГБОУ ВО Вятский государственный университет, кафедра фундаментальной математики, г. Киров}
\practicsbase{ИП Карпов А.А., г. Киров}


\usepackage{longtable,array, hhline}

\renewcommand{\theFancyVerbLine}{\footnotesize\arabic{FancyVerbLine}}


\newcommand{\TODO}[1]{{\color{red} \textbf{TODO: }#1}}
\sloppy


\begin{document}

\maketitle
\newpage

\tableofcontents

\Chapter{Введение}

\paragraph{Вид практики:}	производственная практика.

\paragraph{Тип практики:} научно-исследовательская работа.

\paragraph{Сроки прохождения практики:}
	c~2 октября 2023 года по 24 декабря 2023 года.

\paragraph{Место прохождения практики:}
ИП Карпов А.А., г. Киров



\paragraph{Цель практики:}
Формирование  представления о научной деятельности и развитие интереса к профессиям ученого-исследователя, преподавателя вуза, IT-специалиста.

\paragraph{Задачи практики:}~\par
	\begin{enumerate}
		\item  самостоятельная разработка научно-исследовательского проекта.
		\item  закрепление теоретических знаний, полученных в ходе обучения по направлению подготовки;
		\item  применение методов математического и алгоритмического моделирования при анализе прикладных проблем;
		\item  применение численных и символьных методов при решении математических задач, возникающих в научной, производственной и технологической деятельности;
		\item  адаптация к исследовательской и производственной деятельности.
	\end{enumerate}




\chapter{Ход практики}

\section{Общая характеристика задач решаемых в период практики}
\subsection{Характеристика сферы деятельности предприятия}
Разработка компьютерного программного обеспечения (62.01),
Деятельность консультативная и работы в области компьютерных технологий (62.02)
Научные исследования и разработки в области естественных и технических наук прочие (72.19)
Научные исследования и разработки в области общественных и гуманитарных наук (72.2)
\subsection{Описание задачи, поставленной руководителем от профильной организации}
Найти методику, которая позволила бы находить изменения в характере поступающих данных,
что непосредственно связано с деятельностью предприятия.

\section{Хронологический аннотированный перечень выполненных работ за период
производственной практики}
	В соответствии с индивидуальным заданием в ходе производственной практики выполнены следующие работы:

{\centering
\noindent\begin{longtable}{|>{\centering}p{1cm}|p{11.5cm}|>{\centering}p{3cm}|}
\hline
\textbf{№\newline п/п} &
\centering\textbf{Перечень работ, выполненных в ходе практики} &
\textbf{Сроки выполнения} \tabularnewline
\hline
1 &
Ознакомление с правилами внутреннего трудового распорядка; прохождение инструктажа по охране труда, технике безопасности, противопожарной безопасности, санитарно-эпидемиологическими правилами и гигиеническими нормативами, а также вводного инструктажа и инструктажа на рабочем месте &
02.10.2023 \tabularnewline
\hline
2 &
Ознакомление со сферой деятельности предприятия &
с~7.10.2023
по~21.10.2023 \tabularnewline
\hline
3 &
Изучение поставленных задач
\begin{itemize}[leftmargin=2em]
\item характеристика задач, поставленных руководителем от профильной организации,
\item выявление взаимосвязей решаемых задач с другими задачами предприятия.
\end{itemize} &
23.10.2023,
28.10.2023 \tabularnewline
\hline
4 &
Выполнить подбор и анализ научной и научно-технической литературы по разрабатываемой задаче, подготовить обзор научной и технической литературы &
30.10.2023, 4.11.2023\tabularnewline
\hline
5 &
Выполнение исследовательского проекта (задания от предприятия):
\begin{itemize}[leftmargin=2em]
\item описание необходимое техническое и программное обеспечение для решения поставленной задачи;
\item описание математических методы решения поставленной задачи.
\item выполнение поставленной задачи;
\item описание результатов решения поставленной задачи.
\end{itemize} &
с~7.11.2023 по~18.12.2023 \tabularnewline
\hline
6 &
Подготовка отчета по практике &
16.12.2023, 18.12.2023\tabularnewline
\hline
7 &
Доклад на итоговой конференции по практике&
23.12.2023\tabularnewline
\hline
\end{longtable}
\par}



\section{Краткий обзор научной и научно-технической литературы}
Для решения поставленной задачи изучена научно и научно-техническая литературы, перечень приведен в библиографическом списке на стр.~\pageref{sec:bib}.

\section{Перечень использованного программного обеспечения  для решения поставленных задач}
	\label{sec:Soft}
	\begin{enumerate}
	\item
		Издательская система \LaTeX{} URL: \verb"https://miktex.org"
	\item
		Дистрибутив \LaTeX, \textbf{texlive}  URL: \verb"https://tug.org/texlive"
	\end{enumerate}

\chapter{Индивидуальное задание}

\section{Формализация постановки задачи}
Необходима методика, которая поможет ответить на вопросы
	\begin{enumerate}
	\item
		Как подсветить те показатели, которые изменились больше остальных
		и как отразить это в аналитических ашбордах?
	\item
		Как найти факторы, которые приводят к повышенным значениям целевой переменной?
		(поиск факторов возникновения дефектов)
	\item
		Как при выполнении разведочного анализа данных провести анализ
		изменения данных в обучающих выборках?
	\item
		Как проверить, как отражаются изменения в реальном мире на
		исходные данные и исходные гипотезы модели?
	\item
		С помощью каких показателей построить систему мониторинга качества прогнозов
		моделей машинного обучения, работающих в реальных условиях?
	\end{enumerate}

\section{Математические методы, применяемые для решения поставленных задач}
Численное дифференцирование, STL-декомпозиция.
\section{Структура решения поставленной задачи}
Решение заключается в том, чтобы взять набор данных, найти его конечную разность и применить
алгоритм поиска аномалий (мною выбран метод STL-декомпозиции).
\section{Достигнутые результаты}
Получена работающая методика, отвечающая требованиям руководителя. Изучен ряд методов для поиска аномалий.
Код, написанный для решения задачи, можно найти по URL: "https://github.com/faradayawerty/uni/tree/main/sem5\_mk/prac"




\Chapter{Заключение}
Производственная практика №\,1 (научно-исследовательская работа) проходила на базе ВятГУ c~2 октября 2023 года по 24 декабря 2023 года в г. Кирове.

Основным результатом прохождения практики стала разработанная
математическая модель для поиска точки, в которой меняется поведение
набора данных.

В ходе производственной практики:

\begin{itemize}
	\item \textbf{получены знания} из области дискретного анализа, анализа данных.
	\item \textbf{сформированы умения} находить конечные разности и аномалии в наборе данных.
	\item \textbf{освоена технология} STL, позволяющая искать аномалии методом декомпозиции на сезон-тренд.
	\item \textbf{приобретен опыт} разработки на Python и поиска аномалий с использованием библиотеки STL.
	\item \textbf{собрана информация} о конечных разностях и поиске аномалий.
\end{itemize}








\begin{thebibliography}{9}
\label{sec:bib}

\bibitem{PythonDocumentation}
Руководство по Python. 2001--2023 URL: \textbf{https://docs.python.org/3/} (дата обращения 23.12.2023)

\bibitem{Python Numerical Methods}
Python Numerical Methods 2020--2023 URL: \textbf{https://pythonnumericalmethods.berkeley.edu/
notebooks/chapter23.03-Finite-Difference-Method.html}
(дата обращения 23.12.2023)

\bibitem{Google Trends}
Google Trends for Minecraft searches URL: \textbf{"https://trends.google.com/trends"}
(дата обращения 23.12.2023)

%\bibitem{Gitis}
%Гитис~Л.\,Х. Кластерный анализ в задачах классификации, оптимизации и прогнозирования.~--- Московский государственный горный университет, 2001.~--- 104~с.

\end{thebibliography}
\end{document}

