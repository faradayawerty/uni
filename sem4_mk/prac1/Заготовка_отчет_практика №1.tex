
\documentclass[14pt,Report]{diplomwork}
\reporttype{по учебной практике №\,2}

\date{2020}
\author[Заботин М.\,А.]{МКб-3301-51-00}{Заботин Михаил Алексеевич}
\advisor[Д.\,В. Чупраков]{канд. физ.-матем. наук, доцент}{Д.\,В. Чупраков}
\napravlenie{02.03.01}{Математика и компьютерные науки}
\profile{Математические основы компьютерных наук}
\institute{математики и информационных систем}
\department{компьютерных и физико-математических наук}{Н.\,А.~Бушмелева}
\kafedra[ФМ]{фундаментальной математики}{Е.\,М.~Вечтомов}
\keywords{УЧЕБНАЯ ПРАКТИКА, LATEX}
\annotation{
	В ходе практики решались задачи составления технического задания;
	определения актуальности темы исследования;
	постановки цели и задач исследования;
	организации деятельности в малой исследовательской группе;
	cбора и анализа литературы по теме исследования: Кватернионы.
}

\usepackage{amssymb}
\usepackage{tikz}
\sloppy

\begin{document}
	\maketitle
	\makereferat
	\tableofcontents
	\Chapter{Введение}
		\paragraph{Тип практики:}
		Научно-исследовательская работа .
		\paragraph{Форма практики:} Стационарная.
		\paragraph{Сроки прохождения практики:}
			c 27 июня 2023 по 10 июля 2023г.
		\paragraph{Место прохождения практики:}
			Федеральное государственное бюджетное
			образовательное учреждение высшего образования
			<<Вятский государственный университет>>.
			Кафедра фундаментальной математики.
			\paragraph{Цель практики:}
				получение первичных навыков
				научно-исследовательской работы:
				выполнения расчетных работ,
				оформления результатов научно-исследовательской работы,
				оформления списка литературы.

			\paragraph{Задачи практики:}
			\begin{enumerate}
				\item
					закрепление и углубление теоретических знаний
					и отработка практических навыков по прослушанным
					за время обучения в университете дисциплинам; 
				\item
					знакомство с современными информационными
					технологиями поддержки  научных исследований
				\item
					формирование умений и навыков оформления
					результатов научной работы;
				\item
					формирование  базовых профессиональных навыков и
					умений в области выполнения научно-исследовательской работы
				\item
					формирование информационной компетентности с
					целью успешной профессиональной деятельности;
			\end{enumerate}

			\paragraph{Руководитель практики от организации:}
			Чупраков Дмитрий Вячеславович, канд. физ.-матем. наук, доцент
			\chapter{Производственное задание}
			\section{Общая характеристика задач, решаемых в период практики}
			\label{ZUN}
			В ходе практики решались проводились следующие работы:
			\begin{enumerate}
			\item
				Изучение оформления списка литературы в соответствии с ГОСТ Р7.0.5-2008~\cite{Gost_R_7.0.5-2008_Bib}
			\item
				Изучение оформления научного отчета в соответствии с ГОСТ 7.32-2017 \cite{Gost_7.32-2017_NIR}  и СТП ВятГУ 101-2004~\cite{STP-101-2004}
			\item
				Изучение издательской системы \TeX.
			\item
				Тренировка по публичному выступлению и защите выполненной работы в форме командного соревнования <<Математический бой>>
			\item
				Подготовка отчета по результатам выполненных работ
			\end{enumerate}

			\section{Перечень использованного программного обеспечения}
			Для выполнения заданий практики использовалось следующее следующее программное обеспечение:
			\begin{enumerate}
			\item
				Издательская система \LaTeX, входящая в дистрибутив \TeX Live.
				URL: \url{https://www.tug.org/texlive/}
				(свободные лицензии GNU GPL~2 и LaTeX Project Public License)
			\end{enumerate}
			\section{Хронологический аннотированный перечень выполненных работ}
				\begin{description}
				\item[28.06.2021] Инструктаж по технике безопасности на рабочем месте и пожарной безопасности. Обсуждение индивидуального задания. Изучение рабочего графика и программы практики.
				\item[28.06.2021] Изучение методических рекомендаций по прохождению практики. Установка необходимого программного обеспечения. 
				\item[11.07.2021] Прохождение промежуточной аттестации. Размещение отчетов в система Мoodle ВятГУ.
			\end{description}
			\chapter{Кватернионы}
				\section{Мотивация, история появления}

					Система кватернионов была впервые опубликована Гамильтоном в 1843 году. Историки науки также обнаружили наброски по этой теме в неопубликованных рукописях Гаусса, относящихся к 1819—1820 годам. Также кватернионы рассматривал Эйлер. Б. О. Родриг (1840 год) при рассмотрении поворотов абсолютно твёрдого тела вывел правила умножения кватернионов.

					Бурное и чрезвычайно плодотворное развитие комплексного анализа в XIX веке стимулировало у математиков интерес к следующей задаче: найти новый вид чисел, аналогичный по свойствам комплексным, но содержащий не одну, а две мнимые единицы. Предполагалось, что такая модель будет полезна при решении пространственных задач математической физики. Однако работа в этом направлении оказалась безуспешной.

					Новый вид чисел был обнаружен ирландским математиком Уильямом Гамильтоном (который также занимался указанной задачей) в 1843 году, и он содержал не две, как ожидалось, а три мнимые единицы. Гамильтон работал сначала с дуплетами (точками на плоскости) и легко получил правила для умножения соответствующие комплексным числам, но для точек в пространстве (триплеты) не мог получить никакой формулы умножения для таких наборов. В конце концов решил попробовать четвёрки — точки в четырёхмерном пространстве. Эти числа Гамильтон назвал кватернионами. Позднее Фробениус строго доказал (1877) теорему, согласно которой расширить комплексное поле до поля или тела с двумя мнимыми единицами невозможно.

					Развитие кватернионов и их приложений в физике следовало по трём путям, связанным с алгебраическим подходом, апологетами которого выступали Кэли, который в 1858 году открыл матричное представление кватернионов, Клиффорд, Б. Пирс, Ч. Пирс и Фробениус; с теорией комплексных кватернионов, представителями которого были Клиффорд, Штуди и Котельников; с физикой из-за имён Максвелла и Хэвисайда. Несмотря на необычные свойства новых чисел (их некоммутативность), эта модель довольно быстро принесла практическую пользу. Максвелл использовал компактную кватернионную запись для формулировки своих уравнений электромагнитного поля. Позднее на основе алгебры кватернионов был создан трёхмерный векторный анализ (Гиббс, Хевисайд). Применение кватернионов было вытеснено векторным анализом из уравнений электродинамики. Впрочем тесная связь уравнений Максвелла с кватернионами не исчерпывается только электродинамикой, поскольку формулировка СТО в терминах 4-векторов Минковским была построена теория СТО с использованием кватернионов А. У. Конвеем.

				\section{Алгебраическая структура кватернионов}
					Кватернионы — система гиперкомплексных чисел, образующая векторное пространство размерностью четыре над полем вещественных чисел. Обычно обозначаются символом $\mathbb{H}$.
					\par
					Кватернионы можно определить как сумму $q = a + bi + cj + dk$, где $a,b,c,d \in \mathbb{R}$, а $i,j,k$ - мнимые единицы со следующим свойством: $i^2 = j^2 = k^2 = ijk = 1$. Множество кватернионов является примером тела, то есть кольца с делением и единицей. Множество кватернионов образует четырёхмерную ассоциативную алгебру с делением над полем вещественных (но не комплексных) чисел. По теореме Фробениуса тела 
${\mathbb{R}}, {\mathbb{C}}, {\mathbb{H}}$ являются единственными конечномерными ассоциативными алгебрами с делением над полем вещественных чисел.
					\par
					Некоммутативность умножения кватернионов приводит к неожиданным последствиям. Например, количество различных корней полиномиального уравнения над множеством кватернионов может быть больше, чем степень уравнения. В частности, уравнение $q^{2}+1=0$ имеет бесконечно много решений — это все единичные чисто векторные кватернионы.
					\par
					Для кватерниона $q$ сопряжённым называется: $\bar {q}=a-bi-cj-dk$ Сопряжённое произведение есть произведение сопряжённых в обратном порядке: ${\bar{pq}}=\bar{q}\bar{p}$ Для кватернионов справедливо равенство ${\overline {p}}=-{\frac {1}{2}}(p+ipi+jpj+kpk)$



Так же, как и для комплексных чисел,
$\displaystyle \left|q\right|={\sqrt {q{\bar {q}}}}={\sqrt {a^{2}+b^{2}+c^{2}+d^{2}}}$
называется модулем 
$\displaystyle q$. Если 
$\displaystyle \left|q\right|=1,$ то 
$\displaystyle q$ называется единичным кватернионом.
В качестве нормы кватерниона обычно рассматривают его модуль: 
$\displaystyle \left\|z\right\|=\left|z\right|$.
Таким образом, на множестве кватернионов можно ввести метрику. Кватернионы образуют метрическое пространство, изоморфное 
$\displaystyle \mathbb {R} ^{4}$ с евклидовой метрикой.
Кватернионы с модулем в качестве нормы образуют банахову алгебру.
Из тождества четырёх квадратов вытекает, что 
$\displaystyle \left|p\cdot q\right|=\left|p\right|\cdot \left|q\right|,$ иными словами, кватернионы обладают мультипликативной нормой и образуют ассоциативную алгебру с делением.


Кватернион, обратный по умножению к 
$\displaystyle q$, вычисляется так: 
$\displaystyle q^{-1}={\frac {\bar {q}}{\left|q\right|^{2}}}$.



Кватернионы, рассматриваемые как алгебра над 
$\displaystyle \mathbb {R} $, образуют четырёхмерное вещественное векторное пространство. Любой поворот этого пространства относительно 
$\displaystyle 0$ может быть записан в виде 
$\displaystyle q\mapsto \xi q\zeta $, где 
$\displaystyle \xi $ и 
$\displaystyle \zeta $ — пара единичных кватернионов, при этом пара 
$\displaystyle \left(\xi ,\zeta \right)$ определяется с точностью до знака, то есть один поворот определяют в точности две пары — 
$\displaystyle \left(\xi ,\zeta \right)$ и 
$\displaystyle \left(-\xi ,-\zeta \right)$. Из этого следует, что группа Ли 
$\displaystyle {\text{SO}}\left(\mathbb {R} ,4\right)$ поворотов 
$\displaystyle \mathbb {R} ^{4}$ есть факторгруппа 
$\displaystyle S^{3}\times S^{3}/\mathbb {Z} _{2}$, где 
$\displaystyle S^{3}$ обозначает мультипликативную группу единичных кватернионов. Чисто векторные кватернионы образуют трёхмерное вещественно векторное пространство. Любой поворот пространства чисто векторных кватернионов относительно 
$\displaystyle 0$ может быть записан в виде 
$\displaystyle u\mapsto \xi u{\bar {\xi }}$, где 
$\displaystyle \xi $ — некоторый единичный кватернион. Соответственно, 
$\displaystyle {\text{SO}}\left(\mathbb {R} ,3\right)=S^{3}/\mathbb {Z} _{2}$, в частности, 
$\displaystyle {\text{SO}}\left(\mathbb {R} ,3\right)$ диффеоморфно 
$\displaystyle \mathbb {R} \mathrm {P} ^{3}$.
				\section{Приложения кватернионов}
В XX веке были сделаны несколько попыток использовать кватернионные модели в квантовой механике и теории относительности. Реальное применение кватернионы нашли в современной компьютерной графике и программировании игр, а также в вычислительной механике, в инерциальной навигации и теории управления. С 2003 года издаётся журнал «Гиперкомплексные числа в геометрии и физике».

Во многих областях применения были найдены более общие и практичные средства, чем кватернионы. Например, в наши дни для исследования движений в пространстве чаще всего применяется матричное исчисление. Однако там, где важно задавать трёхмерный поворот при помощи минимального числа скалярных параметров, использование параметров Родрига — Гамильтона (то есть четырёх компонент кватерниона поворота) весьма часто оказывается предпочтительным: такое описание никогда не вырождается, а при описании поворотов тремя параметрами (например, углами Эйлера) всегда существуют критические значения этих параметров, когда описание вырождается.

Как алгебра над 
$\displaystyle \scriptstyle \mathbb {R} $, кватернионы образуют вещественное векторное пространство 
$\displaystyle \scriptstyle \mathbb {H} $, снабжённое тензором третьего ранга 
$\displaystyle S$ типа (1,2), иногда называемого структурным тензором. Как всякий тензор такого типа, 
$\displaystyle S$ отображает каждую 1-форму 
$\displaystyle t$ на 
$\displaystyle \scriptstyle \mathbb {H} $ и пару векторов 
$\displaystyle \left(a,b\right)$ из 
$\displaystyle \scriptstyle \mathbb {H} $ в вещественное число 
$\displaystyle S\left(t,a,b\right)$. Для любой фиксированной 1-формы 
$\displaystyle t$
$\displaystyle S$ превращается в ковариантный тензор второго ранга, который, в случае его симметрии, становится скалярным произведением на 
$\displaystyle \scriptstyle \mathbb {H} $. Поскольку каждое вещественное векторное пространство является также вещественным линейным многообразием, такое скалярное произведение порождает тензорное поле, которое, при условии его невырожденности, становится (псевдо- или собственно-)евклидовой метрикой на 
$\displaystyle \scriptstyle \mathbb {H} $. В случае кватернионов это скалярное произведение индефинитно, его сигнатура не зависит от 1-формы 
$\displaystyle t$, а соответствующая псевдоевклидова метрика есть метрика Минковского. Эта метрика автоматически продолжается на группу Ли ненулевых кватернионов вдоль её левоинвариантных векторных полей, образуя так называемую закрытую ФЛРУ (Фридман — Леметр — Робертсон — Уолкер) метрику — важное решение уравнений Эйнштейна. Эти результаты проясняют некоторые аспекты проблемы совместимости квантовой механики и общей теории относительности в рамках теории квантовой гравитации.






			\chapter{Заключение}
			Учебная практика проходила в ФГБОУ ВО <<Вятский государственный университет>>
			с 27.06.2023 по 10.07.2023.
			В ходе учебной практики освоены технологии, описанные в параграфе~\ref{ZUN}.
			Получены знания об истории и природе кватернионов, их алгебраической структуре и об их приложениях.
			%TODO Указать какие
			\begin{thebibliography}{99}
			\bibitem{Chupr}
				Соколова А.\,Н., Чупраков Д.\,В. Оформление результатов исследовательской работы студентов в LATEX : учеб. пособие для студентов вузов / ВятГГУ. Киров : Радуга-ПРЕСС, 2013. --- 256 с
			\bibitem{Chupr}
				Кантор И.Л., А.С. Солодовников Гиперкомплексные числа
			\bibitem{Chupr}
				Побегайло А.П. Применение кватернионов в компьютерной геометрии и графике
			\end{thebibliography}
			%\APPENDIX
			%\chapter{Какое-то приложение}
\end{document}

